% Chapter Template

\chapter{Implementing the PO-TITE-CRM trial design into ADePT-DDR} % Main chapter title

\label{Chapter1} % Change X to a consecutive number; for referencing this chapter elsewhere, use \ref{ChapterX}

\section{Draft Structure}
\begin{itemize}
	\item Introduction 
		\begin{itemize}
			\item Basic biological background 
			\item Main objective of the trial 
			\item Introduce the new TD notation link to previous trial designs 
			\item Paragraph on traditional dose finding trial designs 3+3, CRM etc. 
			\item Methodological issues which arise due to investigating combination of drugs/varying parameters (new concept by Piers)
			\item Necessity of time-to-event components for DLTs which may occur later 
			\item Other possible methodologies in this area which may be of use to solve this problem
			\item Mini literature search will do a citation search for both methodology papers (potentially use a table/figure to summarise) 
			\item Detail whats to come in the chapter
		\end{itemize}
	\item The PO-TITE-CRM Design
	\item PO-TITE-CRM in ADePT-DDR 
	\item Modifications to the specification to improve operating characteristics 
	
\end{itemize}

%----------------------------------------------------------------------------------------
%	SECTION 1
%----------------------------------------------------------------------------------------

\section{Introduction}

Worldwide there are approximately 600,000 new cases of Head and Neck Squamous Cell Carcinoma (HNSCC) each year \cite{stransky_mutational_2011}. Of which, 12,000 occur in the UK with the most common forms of treatment being surgery, radiotherapy and/or \cite{cancer_research_uk_head_2017}. Radiotherapy is essential for the treatment of cancer, it has been estimated that more than 40\% of patients will receive radiotherapy at some point in their treatment \cite{round_radiotherapy_2013}. However, despite recent advancements in radiation techniques and the use of of concomitant chemo radiotherapy, patients with solid tumours such as head and neck cancer have suboptimal cure rates \cite{cancer_research_uk_head_2017,cognetti_head_2008}. For those with advance HNSCC primary radiotherapy with concurrent chemotherapy is often offered but, it has not been shown to improve survival in patients aged over 70 compared to radiotherapy alone \cite{pignon_chemotherapy_2000}. Therefore, any strategy to improve the efficacy of radiotherapy without increasing toxicity to normal tissue would have a significant impact for patients. DNA damage repair (DDR) inhibition is a potential technique which could be utilised as it potentiates the therapeutic effects of ionising radiation in cancer cells without a substantial increase in acute and late toxicity. Combining radiotherapy with DDR inhibition could improve clinical outcomes for these patients \cite{chalmers_science_2016}.  


%-----------------------------------
%	SUBSECTION 1
%-----------------------------------
\subsection{Subsection 1}

Nunc posuere quam at lectus tristique eu ultrices augue venenatis. Vestibulum ante ipsum primis in faucibus orci luctus et ultrices posuere cubilia Curae; Aliquam erat volutpat. Vivamus sodales tortor eget quam adipiscing in vulputate ante ullamcorper. Sed eros ante, lacinia et sollicitudin et, aliquam sit amet augue. In hac habitasse platea dictumst.

%-----------------------------------
%	SUBSECTION 2
%-----------------------------------

\subsection{Subsection 2}
Morbi rutrum odio eget arcu adipiscing sodales. Aenean et purus a est pulvinar pellentesque. Cras in elit neque, quis varius elit. Phasellus fringilla, nibh eu tempus venenatis, dolor elit posuere quam, quis adipiscing urna leo nec orci. Sed nec nulla auctor odio aliquet consequat. Ut nec nulla in ante ullamcorper aliquam at sed dolor. Phasellus fermentum magna in augue gravida cursus. Cras sed pretium lorem. Pellentesque eget ornare odio. Proin accumsan, massa viverra cursus pharetra, ipsum nisi lobortis velit, a malesuada dolor lorem eu neque.

%----------------------------------------------------------------------------------------
%	SECTION 2
%----------------------------------------------------------------------------------------

\section{Main Section 2}

Sed ullamcorper quam eu nisl interdum at interdum enim egestas. Aliquam placerat justo sed lectus lobortis ut porta nisl porttitor. Vestibulum mi dolor, lacinia molestie gravida at, tempus vitae ligula. Donec eget quam sapien, in viverra eros. Donec pellentesque justo a massa fringilla non vestibulum metus vestibulum. Vestibulum in orci quis felis tempor lacinia. Vivamus ornare ultrices facilisis. Ut hendrerit volutpat vulputate. Morbi condimentum venenatis augue, id porta ipsum vulputate in. Curabitur luctus tempus justo. Vestibulum risus lectus, adipiscing nec condimentum quis, condimentum nec nisl. Aliquam dictum sagittis velit sed iaculis. Morbi tristique augue sit amet nulla pulvinar id facilisis ligula mollis. Nam elit libero, tincidunt ut aliquam at, molestie in quam. Aenean rhoncus vehicula hendrerit.