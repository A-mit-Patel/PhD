% Chapter Template

\chapter{Implementing the PO-TITE-CRM trial design into ADePT-DDR} % Main chapter title

\label{Chapter2} % For referencing this chapter elsewhere, use \ref{Chapter2}

%----------------------------------------------------------------------------------------
%	SECTION 1
%----------------------------------------------------------------------------------------

\section{Introduction}

Worldwide there are approximately 600,000 new cases of Head and Neck Squamous Cell Carcinoma (HNSCC) each year \cite{stranskyMutationalLandscapeHead2011}. Of which, 12,000 occur in the UK with the most common forms of treatment being surgery, radiotherapy and/or chemotherapy \cite{cancerreaserchukHeadNeckCancers2017}. Radiotherapy is essential for the treatment of cancer, it has been estimated that more than 40\% of patients will receive radiotherapy at some point in their treatment \cite{roundRadiotherapyDemandActivity2013}. However, despite recent advancements in radiation techniques and the use of of concomitant chemo radiotherapy, patients with solid tumours such as head and neck cancer have suboptimal cure rates \cite{cancerreaserchukHeadNeckCancers2017,cognettiHeadNeckCancer2008}. For those with advance HNSCC primary radiotherapy with concurrent chemotherapy is often offered but, it has not been shown to improve survival in patients aged over 70 compared to radiotherapy alone \cite{pignonChemotherapyAddedLocoregional2000}. Therefore, any strategy to improve the efficacy of radiotherapy without increasing toxicity to normal tissue would have a significant impact for patients. DNA damage repair (DDR) inhibition is a potential technique which could be utilised as it potentiates the therapeutic effects of ionising radiation in cancer cells without a substantial increase in acute and late toxicity. Combining radiotherapy with DDR inhibition could improve clinical outcomes for these patients \cite{chalmersScienceFocusCombining2016}.  

The ADePT-DDR trial is a platform trial which aims to evaluate the safety and efficacy of different DDR agents, or different immunotherapy agents and/or DDR and immunotherapy combinations, together with radiotherapy in patients with HNSCC. The initial component of this trial is a single arm dose-finding phase \RN{1}b/\RN{2}a trial investigating the Ataxia Telangiectasis and Rad3 Related (ATR) inhibitor AZD6738 in combination with radiotherapy. ATR inhibitors block DNA repair and AZD6738 has been shown to effectively kill cancer cells in preclinical models \cite{meiAtaxiaTelangiectasiaRad3related2019}. 

Traditionally dose-finding trials aim to determine the maximum tolerated dose (MTD) of a treatment based on the cytotoxic assumption that the most toxic dose is the most efficacious. Rule-based or 'up and down' designs achieve this by escalating and de-escalating dose dependent on the observation of severe toxicity due to the drug,  commonly referred to as a dose limiting toxicity (DLT). In the case of the 3+3 design escalation continues till at least two patients in a cohort of three or six experience a DLT. More explicitly, the MTD is the dose at which $\geq$33\% of patients experience a DLT \cite{letourneauDoseEscalationMethods2009}. Model based designs such as the continual reassessment method (CRM) \cite{oquigleyContinualReassessmentMethod1990} work on a slightly different principle which assumes that the probability of toxicity monotonically increases with dose. One key difference with the CRM is that it iteratively changes dose, seeking some acceptable target probability of toxicity also referred to as the MTD. 

Due to the historical use of rule-based designs the majority of terminology used to describe them, and the ambiguity they raise, have been inherited by modern designs such as the CRM. The MTD in the context of a CRM is not the 'maximum' dose patients could tolerate but rather a dose which there would be an acceptable target probability of a DLT occurring. For example, if the target is set at 25\% the MTD would be the dose at which there is a 25\% probability of experiencing a DLT. Rather than using MTD the dose to be found will be referred to as the target dose (TD\%\%, where the \%'s are replaced by the target probability), i.e. TD25 would be the dose expected to be toxic in 25\% of patients.

The investigation of multiple-agent treatments, where the monotonicity assumption may not hold, is increasing in early phase trials. Finding the TD in combinations of treatments, compared to single-agents,  presents methodological challenges. Each drug individually may obey the monotonicity assumption, however, when combined, the ordering of  doses in terms of toxicity may not be fully apparent. An order for a subset of the combined doses could be identified resulting in a partial order. Without a fully understood ordering it is uncertain which dose should be chosen in decisions of escalation and de-escalation and ultimately as the TD. This issue is not exclusively reserved to just trials for multiple-agents. The monotonicity assumption may not hold for certain drugs in single-agent studies leading to partial orders of dose toxicity. This can occur in scenarios where multiple parameters of the treatment schedule are altered for each dose level. For example, either dose or treatment duration could be increased and even if patients  receive an equal dose it would remain unclear as to if prolonged exposure to a lower dose is more toxic than short exposure to a higher dose, which would lead to a partial ordering. 

Further methodological challenges revolve around the issue of late onset toxicities. Typically, early phase trials implement a short window post treatment to observe DLTs. This works well in situations where toxicities are likely to occur rapidly after treatment. However, this is not optimal for treatments which could cause late-onset toxicities such as radiotherapy. The aim here would be to incorporate a larger observation window to account for potential late onset toxicities whilst also minimising the trial duration. 

Cheung and Chappel \cite{cheungSequentialDesignsPhase2000} introduced an extension to the CRM to deal with the issues of treatments that may cause late-onset toxicity. This design, referred to as the time-to-event CRM (TITE-CRM), uses a weighted dose response model to incorporate the time it takes for a DLT to occur in a patient. There have also been published trial designs to deal with the issues that arise from investigating combinations of treatments. Thall et al. \cite{thallDoseFindingTwoAgents2003} proposed an adaptive two-stage Bayesian design which utilises a parametric model of toxicity as a function of two doses. Yin and Yaun \cite{yinBayesianDoseFinding2009} present a Bayesian design which uses a copula regression model to evaluate the joint toxicity probabilities of combined drug. The continual reassessment method for partial orders (PO-CRM) developed by Wages et al. \cite{wagesContinualReassessmentMethod2011} extends the CRM design by relaxing the assumption of monotonicity and by modelling different potential orders.  

Wages et al. \cite{wagesContinualReassessmentMethod2011, wagesUsingTimetoeventContinual2013} further developed their work on the PO-CRM to deal with late-onset toxicities by implementing a TITE component. This trial design, referred to as the time-to-event continual reassessment method in the presence of partial orders (PO-TITE-CRM) by the authors, was chosen to be used ADePT-DDR. A search of PubMed, conducted on the 25th of July 2020, found six articles that had cited the PO-TITE-CRM design by Wages et al. \cite{wagesUsingTimetoeventContinual2013}. Five of these papers were methodological in nature, two of which only include the PO-TITE-CRM design in a brief introduction to current methodology before going on to present new Bayesian trial designs \cite{liuBAYESIANDATAAUGMENTATION2013, wheelerBayesianModelfreeApproach2019}. The other three papers were authored by Wages himself. The first of which details practical considerations and specifications for the PO-CRM design, the TITE variant is only cited as the source of an example which is being used \cite{wagesSpecificationsContinualReassessment2013}. One paper presents an R package 'pocrm' \cite{wagesPocrmRpackagePhase2013}, the package is only capable of analysing the PO-CRM design, similarly the TITE variant is only referenced as it illustrates the issue of partial ordering. The last methodological paper by Wages et al. \cite{wagesPracticalDesignsPhase2016} presents three different methods for phase \RN{1} studies of drug combinations one of which is the PO-CRM however, PO-TITE-CRM is only mentioned as an extension to this design. A key message in this paper is the fact that novel methodologies are constantly emerging but are rarely implemented in practice. The last paper is a protocol paper which is investigating a combination of treatments however, it is only escalating dose in one of the drugs \cite{lesueurPhaseIIaStudy2019}. The PO-TITE-CRM paper is cited but it appears as if another variant of the TITE-CRM is being used. This is just a brief review of the current literature but it seems that the PO-TITE-CRM has rarely been used or discussed since its inception. 

Section \ref{section2.2} will detail how the PO-TITE-CRM works. Section \ref{section2.3} discusses the justification for implementing the design into the ADePT-DDR trial and our experiences doing so.  

%----------------------------------------------------------------------------------------
%	SECTION 2
%----------------------------------------------------------------------------------------
\section{The PO-TITE-CRM Design}
\label{section2.2} %first number is the chapter number, second number is section number, third is subsection etc...

%----------------------------------------------------------------------------------------
%	SECTION 3
%----------------------------------------------------------------------------------------

\section{PO-TITE-CRM in ADePT-DDR}  
\label{section2.3}%first number is the chapter number, second number is section number, third is subsection etc..

%-----------------------------------
%	SUBSECTION 1
%-----------------------------------

%-----------------------------------
%	SUBSECTION 2
%-----------------------------------
