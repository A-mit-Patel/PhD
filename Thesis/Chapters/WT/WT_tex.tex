% Chapter Template

\chapter{Extensions to the Wages and Tait trial design} % Main chapter title

\label{WT} % For referencing this chapter elsewhere, use \ref{WT}

%----------------------------------------------------------------------------------------
%	SECTION 1
%----------------------------------------------------------------------------------------

\section{Plan for this Chapter}

The main plan for this chapter will be as follows. It will comprise of three sections an Introduction / background then two section on the each of the Wages and Tait modifications. One for the randomisation and one for the time-to-event component. 

%-----------------------------------
%	SUBSECTION 1
%-----------------------------------
\subsection{Introduction Plan}

I say Introduction but this can most likely be split into 2 sections. It can serve as an introduction to Phase I/II trials and an introduction to Wages and Tait specifically, which will include the maths. 

\begin{enumerate}
	\item A background on adaptive phase I/II designs - mainly a motivation / rationale for why they are used.
	\item Some examples of these types of designs obviously can use efftox and wages and tait here. Can look at Kristian's PhD here as he has some examples. It may also be worthwhile to do a literature review to look into other designs. 
	\item Include a paragraph stating aims of the chapter or whats included.
	\item Include further details on wages and tait which will then lead to the specific maths for an un modified design.
\end{enumerate}


%-----------------------------------
%	SUBSECTION 2
%-----------------------------------

\subsection{Randomisation Modification Plan}
Not sure what this section should be called. Perhaps something along the lines of Randomisation to control modification. 

\begin{enumerate}
	\item Start off with rational, why we would want to include randomisation to control 
	\item Point out that this could already be done by including a dose as control. However, point out that it is unlikely to recruit or more so you can't guarantee that patients will be allocated that dose as it will be relatively low on efficacy in the AR phase. This is the main distinction with this modification we fix some randomisation. 
	\item Go into the maths. Should only be a small modification in the adaptive randomisation phase. 
	\item Simulations. Need to present a basic design then contrast it something. A quick idea is to have the same trial but then change the percentage that are allocated to placebo. Then contrast this with a trial that does randomisation AT recruitment i.e. 2:1 then enter Wages and Tait. So compare 2:1 to a 33\% fixed, 3:1 to 25\%, 4:1 to a 20\% etc ... Should be easy to make some plots for this. Can also investigate the impact of altering the number of patients in the adaptive randomisation phase, like I did for those quick simulations where I had it set at 26 then 52. 
	\item Observations from running SPIN-SCI simulations. Running a shortened adaptive randomisation phase and front-loading patients onto control seems to result in better operating characteristics when compared to a standard Wages and Tait design. Another comparison to look at hear would be a wages and tait design with a further reduced adaptive randomisation phase. One point to make could be that this front loading could be used in normal wages and tait designs where you want to be extra careful with recruitment and for safety reasons want to fix recruitment at the lowest dose-level before escalating. 
\end{enumerate}


%-----------------------------------
%	SUBSECTION 3
%-----------------------------------

\subsection{Time-to-event Modification Plan}
This should be more straight forward. Will probably have to work more on the code for this one. 

\begin{enumerate}
	\item Again start with rational for why this modification might be useful. Can elude to some stuff mentioned in the Adept chapter. 
	\item How this impacts the maths what equations are altered etc. 
	\item Simulations. Consider a simple trial with a long follow-up period and then simulate using normal wages and tait and then the TITE modification. Can look at the original TITE paper to look how they made comparisons between the designs. May be beneficial to track things like duration also. 
\end{enumerate}


%----------------------------------------------------------------------------------------
%	SECTION 2
%----------------------------------------------------------------------------------------

\section{Introduction}

