% Chapter Template

\chapter{Extensions to the Wages and Tait trial design} % Main chapter title

\label{WT} % For referencing this chapter elsewhere, use \ref{WT}

%----------------------------------------------------------------------------------------
%	SECTION 1
%----------------------------------------------------------------------------------------

\section{Plan for this Chapter}

The main plan for this chapter will be as follows. It will comprise of three sections an Introduction / background then two section on the each of the Wages and Tait modifications. One for the randomisation and one for the time-to-event component. 

%-----------------------------------
%	SUBSECTION 1
%-----------------------------------
\subsection{Introduction Plan}

I say Introduction but this can most likely be split into 2 sections. It can serve as an introduction to Phase I/II trials and an introduction to Wages and Tait specifically, which will include the maths. 

\begin{enumerate}
	\item A background on adaptive phase I/II designs - mainly a motivation / rationale for why they are used.
	\item Some examples of these types of designs obviously can use efftox and wages and tait here. Can look at Kristian's PhD here as he has some examples. It may also be worthwhile to do a literature review to look into other designs. 
	\item Include a paragraph stating aims of the chapter or whats included.
	\item Include further details on wages and tait which will then lead to the specific maths for an un modified design.
\end{enumerate}


%-----------------------------------
%	SUBSECTION 2
%-----------------------------------

\subsection{Randomisation Modification Plan}
Not sure what this section should be called. Perhaps something along the lines of Randomisation to control modification. 

\begin{enumerate}
	\item Start off with rational, why we would want to include randomisation to control 
	\item Point out that this could already be done by including a dose as control. However, point out that it is unlikely to recruit or more so you can't guarantee that patients will be allocated that dose as it will be relatively low on efficacy in the AR phase. This is the main distinction with this modification we fix some randomisation. 
	\item Go into the maths. Should only be a small modification in the adaptive randomisation phase. 
	\item Simulations. Need to present a basic design then contrast it something. A quick idea is to have the same trial but then change the percentage that are allocated to placebo. Then contrast this with a trial that does randomisation AT recruitment i.e. 2:1 then enter Wages and Tait. So compare 2:1 to a 33\% fixed, 3:1 to 25\%, 4:1 to a 20\% etc ... Should be easy to make some plots for this. Can also investigate the impact of altering the number of patients in the adaptive randomisation phase, like I did for those quick simulations where I had it set at 26 then 52. 
	\item Observations from running SPIN-SCI simulations. Running a shortened adaptive randomisation phase and front-loading patients onto control seems to result in better operating characteristics when compared to a standard Wages and Tait design. Another comparison to look at hear would be a wages and tait design with a further reduced adaptive randomisation phase. One point to make could be that this front loading could be used in normal wages and tait designs where you want to be extra careful with recruitment and for safety reasons want to fix recruitment at the lowest dose-level before escalating. 
\end{enumerate}


%-----------------------------------
%	SUBSECTION 3
%-----------------------------------

\subsection{Time-to-event Modification Plan}
This should be more straight forward. Will probably have to work more on the code for this one. 

\begin{enumerate}
	\item Again start with rational for why this modification might be useful. Can elude to some stuff mentioned in the Adept chapter. 
	\item How this impacts the maths what equations are altered etc. 
	\item Simulations. Consider a simple trial with a long follow-up period and then simulate using normal wages and tait and then the TITE modification. Can look at the original TITE paper to look how they made comparisons between the designs. May be beneficial to track things like duration also. 
\end{enumerate}


%----------------------------------------------------------------------------------------
%	SECTION 1
%----------------------------------------------------------------------------------------

\section{Introduction}

Typically the main aim of Phase \RN{1} clinical trials is to identify the maximum tolerated dose (MTD) of the treatment being investigated. The MTD is usually determined under the cytotoxic assumption which assumes the most toxic dose is the most efficacious. With model based designs such as the continual reassessment method (CRM) \cite{oquigleyContinualReassessmentMethod1990} escalation occurs to identify the dose with an associated probability of toxicity based on a pre-defined target. Dose selection and escalation decisions do not consider efficacy rather they are determined based on the occurrence of toxicities. The cytotoxic assumption here implies that rate of efficacy increases monotonically with the probability of toxicity and dose level. Subsequent Phase \RN{2} trials aim to assess the efficacy of the treatment at the recommended dose (MTD). Usually these two phases are conducted independently of each other and as such the ability to share information across the phases is somewhat lost. 

For treatments like chemotherapy which kills all cells including cancer cells the cytotoxic assumption is valid. However, the emergence of modern treatments such as immunotherapy and molecular targeted agents challenges this paradigm. Immunotherapy is a form of treatment which utilises the body's own immune system to fight the cancer. Molecular targeted agents work by interfering with specific molecules responsible for the growth, spread and progression of cancer. The monotonic assumption of dose-toxicity and dose-efficacy may not hold for these new types of treatments. Furthermore, these treatments in general are less toxic than traditional cytotoxic agents such as chemotherapy therefore it is possible the most efficacious dose may occur at a dose level below the MTD \cite{ahnOptimalBiologicalDose2016}. This produces some methodological challenges for dose-finding trials. Instead of trying to identify the MTD the goal would be to determine the optimal biological dose (OBD). Depending on the aims of the trial and the design implemented the definition of the OBD may vary. The OBD could be a dose that provides the maximum probability of efficacy with the probability of toxicity being less than a pre-defined target value, or the dose that has a beneficial trade-off between toxicity and efficacy. In order to determine an optimal dose both toxicity and efficacy outcomes need to be considered, this leads to a need for joint phase \RN{1}/\RN{2} trial designs. Here we will briefly explore some of these designs. 

Braun \cite{braunBivariateContinualReassessment2002} proposed the bivariate continual reassessment method (bCRM), an extension to the CRM which incorporates competing outcomes for both toxicity and disease progression. The design models the probabilities of toxicity and progression independently, it is suggested that either empiric, logistic and hyperbolic tangent functions are used dependent on their biological plausibility. Both outcomes are then combined into a joint distribution which is used to estimate posterior means based on priors and observed data. 

Thall \& Cook \cite{thallDosefindingBasedEfficacytoxicity2004} developed EffTox , a Bayesian adaptive dose-finding trial based on trade-offs between the probabilities of toxicity and efficacy. Marginal probabilities of efficacy and toxicity at each dose are modelled and used with utility contours to determine the desirability of each dose based on posterior probabilities of efficacy and toxicity \cite{brockImplementingEffToxDosefinding2017}. 

Zhou et al. \cite{zhouUtilitybasedBayesianOptimal2019} introduced a Utility-based Bayesian Optimal Interval (U-BOIN) phase \RN{1}/\RN{2} design to identify the OBD. This design is an extension of the Bayesian optimal interval (BOIN) design for phase \RN{1} trials developed by Liu and Yuan \cite{liuBAYESIANDATAAUGMENTATION2013}. U-BOIN jointly models toxicity and efficacy with a multinomial-Dirichlet model and uses a utility function to measure the dose risk-benefit trade-off. The design consists of two seamless stages. Firstly, in stage \RN{1} the BOIN design is used to explore the dose levels and determine a set of admissible doses and collect preliminary efficacy data. In stage \RN{2} posterior estimates of utility for each dose are continuously updated after each cohort this is done using toxicity and efficacy data from both stages. 

Zhang et al. \cite{zhangAdaptiveDosefindingDesign2006} introduced the trivariate CRM (TriCRM) design. The design considers patients to have one of three possible outcomes: no efficacy and toxicity, efficacy without toxicity and toxicity. These outcomes are then modelled using a continuation-ratio model. A Bayesian approach and dose-finding algorithm is then used to identify the OBD similar to the CRM.  

Anathakrishnan et al. \cite{ananthakrishnanExtensionsMTPITEQR2018} produced extensions to the modified Toxicity Probability Interval (mTPI) design by Ji \& Wang \cite{jiModifiedToxicityProbability2013} and Toxicity Equivalent Range (TEQR) design by Blanchard \& Longmate \cite{blanchardToxicityEquivalenceRange2011} to include efficacy outcomes. In both designs isotonic regression is applied to the observed DLT rates at the end of the trial. Dependent on the shapes of the dose-response curves and the underlying response rates isotonic regression is applied on the observed response rates or the differences in observed response rates to determine the optimal dose. 

This chapter revolves around the seamless phase \RN{1}/\RN{2} dose-finding adaptive design by Wages and Tait \cite{wagesSeamlessPhaseII2015}, that we will refer to as the WT design.
This design models toxicity and efficacy independently. To model the probability of efficacy a set of possible efficacy skeletons are considered which would correspond to plausible dose-efficacy relationships. For the class of dose-efficacy models a single parameter model is used similar to the empiric model of the CRM. The authors recommend that ($2n - 1$) efficacy skeletons are specified where $n$ is the number of doses being investigated. Toxicity is modelled using a CRM approach with an empiric model. As such a skeleton for toxicity is also required for this design. The dose-finding operates in two stages the adaptive randomisation (AR) phase and the maximisation phase. In the AR phase patients are adaptively randomised amongst a set of tolerable doses, where probabilities of randomisation to each dose are proportional to their posterior probabilities of efficacy. A pre-defined number of patients enter the AR phase and once recruitment has completed we move to the maximisation phase. In this phase patients are allocated to the dose in the tolerable set which maximises probability of efficacy.  

The incorporation of an AR phase early on into the trial is beneficial since there may be a lack of data in order to rely on decisions made by the maximisation of efficacy probabilities. Also there may be doses that haven't been tested and randomisation allows for information to be collected from these. It also helps avoid getting stuck repeatedly recruiting to the same dose and allows for a more broad understanding of the dose-efficacy and toxicity relationships. One extension we propose is the inclusion of randomisation to a control arm into the design. This would provide a set of patients who receive standard of care to act as controls and allow for comparisons to be made with outcomes from patients receiving the OBD. There is also the added benefit of being able to include standard of care into the models to get a better understanding of the dose-efficacy and toxicity relationships.  

\textcolor{red}{Include paragraph here to specify whats in each section of this chapter.}

 